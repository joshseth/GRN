\documentclass[a4paper,12pt]{article}
\usepackage{amsmath, amssymb}
\begin{document}

\section{Introduction}

The purpose of the following analysis is to establish a general framework to understand how genes interact to produce phenotypes, and to form predictions about how these interactions can evolve given specific conditions and constrains. 

We study a population of varying organisms in a constant and shared environment.  

\section{Definitions}

All organisms are defined by their collection of $n$ genes and the $n^{2}$ possible interactions among these genes. This system, known as a Genetic Regulatory Network (GRN) is represented as a matrix of interaction coefficients:

GRN $:= A \in \mathbb{R}^{n \times n}$ 

An organism's phenotype is simply the vector of its gene products, and determined by an interaction of its GRN with its initial environmental conditions.

Phenotype $:= v_{\infty}^{(A)}$ and initial environmental conditions $:= v_{0}$

Formally, the phenotype is determined by the mapping (via a GRN): 

$A^{\infty}: v_{0} \rightarrow v_{\infty}^{(A)}$ and is computed by iteratively multiplying the initial vector by the GRN until convergence, $A^{\infty}v_{0} = v_{\infty}^{(A)}$

However, not all aspects of an organism's phenotype will increase or decrease its fitness in a given environment. Thus, we are interested in the aspects of an organism's phenotype that the environment will evaluate. 

The phenotype exposed to natural selection is defined as the vector $c_{\infty}^{(A)} \in \mathbb{R}^{m}$.
Note that this vector is in dimension $m$ and is not necessarily the same as the phenotype dimension, $n$. Natural selection can be evaluate any vector of $m$ functions of the phenotype. 

The matrix $C \in \mathbb{R}^{m \times n}$ is a mapping which isolates only the aspects of a phenotype of interest given the environment. 

$C: v_{\infty}^{(A)} \rightarrow c^{(A)}_{\infty}$

The computation is as follows: $Cv_{\infty}^{(A)} = c_{\infty}^{(A)}$

In this model, there exists one 'optimal' phenotype, $c_{*} \in \mathbb{R}^{m}$, which an organism's phenotype is compared to compute fitness. 

Fitness is defined as: 

$e^{-\lVert c_{\infty}^{(A)} - c_{*} \rVert^{2}}$

The 'optimal' GRN, or a GRN's highest-possible fitness is described by the measure $z \in \mathbb{R}$

$ z := inf\{\lVert c_{\infty}^{(A)} - c_{*} \rVert^{2}: A \in \mathbb{R}^{n \times n}\}$

\section{Ideas}

Conjecture 1:

 If $dim{[c_{\infty}^{(A)}]} \geq dim{[c_{*}]}$ then $z = 0\ \forall \ A$. If $dim{[c_{\infty}^{(A)}]} <  dim{[c_{*}]}$, then $z \geq 0$


\end{document}
